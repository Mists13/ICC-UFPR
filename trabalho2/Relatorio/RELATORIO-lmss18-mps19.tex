\documentclass[10pt,a4paper]{article}
\usepackage[utf8]{inputenc}
\usepackage[T1]{fontenc}
\usepackage[portuguese]{babel}
\usepackage{amsmath,amsthm,amsfonts,amssymb}
\usepackage{listings,lstautogobble}
\usepackage[left=1.2in,right=1.2in,top=1.2in,bottom=1.2in]{geometry}
\usepackage{graphicx}
\usepackage{xcolor}
\author{Matheus Pacheco dos Santos e Luzia Millena Santos Silva}
\title{Trabalho 2 - ICC}
\definecolor{verde}{rgb}{0,0.5,0}
\lstset{
	language=C,
	basicstyle=\ttfamily\small,
	keywordstyle=\color{blue},
	stringstyle=\color{verde},
	commentstyle=\color{red},
	extendedchars=true,
	showspaces=false,
	showstringspaces=false,
	numbers=left,
	numberstyle=\tiny,
	breaklines=true,
	backgroundcolor=\color{green!10},
	breakautoindent=true,
	captionpos=b,
	xleftmargin=0pt,
	autogobble=true,
	framesep=0pt,
	tabsize=4,
	inputencoding=utf8,  % Input encoding
	extendedchars=true,  % Extended ASCII
	literate=        % Support additional characters
	{á}{{\'a}}1  {é}{{\'e}}1  {í}{{\'i}}1 {ó}{{\'o}}1  {ú}{{\'u}}1
	{Á}{{\'A}}1  {É}{{\'E}}1  {Í}{{\'I}}1 {Ó}{{\'O}}1  {Ú}{{\'U}}1
	{à}{{\`a}}1  {è}{{\`e}}1  {ì}{{\`i}}1 {ò}{{\`o}}1  {ù}{{\`u}}1
	{À}{{\`A}}1  {È}{{\'E}}1  {Ì}{{\`I}}1 {Ò}{{\`O}}1  {Ù}{{\`U}}1
	{ä}{{\"a}}1  {ë}{{\"e}}1  {ï}{{\"i}}1 {ö}{{\"o}}1  {ü}{{\"u}}1
	{Ä}{{\"A}}1  {Ë}{{\"E}}1  {Ï}{{\"I}}1 {Ö}{{\"O}}1  {Ü}{{\"U}}1
	{â}{{\^a}}1  {ê}{{\^e}}1  {î}{{\^i}}1 {ô}{{\^o}}1  {û}{{\^u}}1
	{Â}{{\^A}}1  {Ê}{{\^E}}1  {Î}{{\^I}}1 {Ô}{{\^O}}1  {Û}{{\^U}}1
	{œ}{{\oe}}1  {Œ}{{\OE}}1  {æ}{{\ae}}1 {Æ}{{\AE}}1  {ß}{{\ss}}1
	{ç}{{\c c}}1 {Ç}{{\c C}}1 {ø}{{\o}}1  {Ø}{{\O}}1   {å}{{\r a}}1
	{Å}{{\r A}}1 {ã}{{\~a}}1  {õ}{{\~o}}1 {Ã}{{\~A}}1  {Õ}{{\~O}}1
	{ñ}{{\~n}}1  {Ñ}{{\~N}}1  {¿}{{?`}}1  {¡}{{!`}}1
	{°}{{\textdegree}}1 {º}{{\textordmasculine}}1 {ª}{{\textordfeminine}}1 
}

\begin{document}
	\maketitle
	
	\section{Matriz de Derivadas Parciais}
	Seja o sistema não linear de instâncias de \textit{Função Tridiagonal de Broyden}:
	\begin{align*}
		\begin{cases}
			(3-2\cdot x_{1})\cdot x_{1}-2\cdot x_{2}+1 \\
			(3-2\cdot x_{2})\cdot x_{2}-x_{1}-2 \cdot x_{3}+1 \\
			(3-2\cdot x_{3})\cdot x_{3}-x_{2} -2 \cdot x_{4} + 1 \\
			(3-2\cdot x_{3})\cdot x_{4}-x_{3} + 1 
		\end{cases}
	\end{align*}
	através desse sistema a matriz de derivadas parciais irá ficar da forma:
	\begin{align*}
		\begin{pmatrix}
			-2x_{1} + 3 & -2 & 0 & 0 \\
			-1 & - 2x_{2} + 3 & - 2 & 0 \\
			0 & - 1 & - 2x_{3} + 3 & -2 \\
			0 & 0 & - 1 & - 2x_{4} + 3 \\
		\end{pmatrix}
	\end{align*}
	ou seja, a matriz de derivadas parciais de um sistema de instâncias de \textit{Função Tridiagonal de Broyden}, é uma matriz tridiagonal para qualquer que seja a dimensão desta matriz. 
	
	Note que podemos guardar essa matriz tridiagonal com apenas três vetores:
	\begin{align*}
		& D_{1} = 
		\begin{bmatrix}
		 -2 & -2  & -2 \\
		\end{bmatrix} & \\
		& D_{2} = 
		\begin{bmatrix}
			-2x_{1} + 3 & - 2x_{2} + 3 & - 2x_{3} + 3 & - 2x_{4} + 3
		\end{bmatrix} & \\
		& D_{3} = 
		\begin{bmatrix}
		-1 & -1  & -1 \\
		\end{bmatrix} & 
	\end{align*}
	note que $ D_{1} $ e $ D_{3} $, tem $ n - 1 $ elementos no vetor, sendo $ n = 3 $.
	
	Sendo assim, podemos descrever uma matriz de derivadas parciais de um sistema não linear de instâncias de \textit{Função Triadiagonal de Broyden} com apenas $ 3 $ vetores da seguinte forma: 
	\begin{align*}
		& D_{1} = 
		\begin{bmatrix}
		-2 & -2  & -2 & \ldots & -2 \\
		\end{bmatrix} & \\
		& D_{2} = 
		\begin{bmatrix}
		-2x_{1} + 3 & - 2x_{2} + 3 & - 2x_{3} + 3 & - 2x_{4} + 3 & \ldots & - 2x_{n} + 3 
		\end{bmatrix} & \\
		& D_{3} = 
		\begin{bmatrix}
		-1 & -1  & -1 & \ldots & -1 \\
		\end{bmatrix} & 
	\end{align*}
	com $ D_{2} $ com $ n $ elementos e $ D_{1} $, $ D_{3} $ com $ n - 1 $ elementos.
	
	Veja que poderíamos otimizar ao máximo pelo fato de que $ D_{1}, D_{2} $ e $ D_{3} $ terem certas "propriedades" como:
	\begin{itemize}
		\item Todo elemento de $ D_{1} $ é igual $ -2 $.
		\item Todo elemento de $ D_{2} $ é da forma $ -2x_{i} + 3 $.
		\item Todo elemento de $ D_{3} $ é igual $ -1 $.
	\end{itemize}

	Mas para fins didáticos orientados pelo professor vamos calcular as três diagonais para termos uma comparação com o que foi otimizado.
	
	Vamos agora mostrar como fizemos a nossa função que gera as matrizes parciais.

	Veja que podemos gerar as três diagonais da seguinte forma, 
	\begin{lstlisting}
	void geraMatTridiagonalDerivParcial(void ****diagonais, SNL sistema) {
  		for (int i = 0; i < sistema.numFuncoes; i++) {
			sprintf(var,"x%d", i+1);
			(*diagonais)[1][i] = evaluator_derivative(sistema.funcoes[i], var);
		}
		
		for (int i = 0; i < sistema.numFuncoes - 1; i++) {
			sprintf(var,"x%d", i+2);
			(*diagonais)[0][i] = evaluator_derivative(sistema.funcoes[i], var);
		}
		
		for (int i = 0; i < sistema.numFuncoes-1; i++) {
			sprintf(var,"x%d", i+1);
			(*diagonais)[2][i] = evaluator_derivative(sistema.funcoes[i+1], var);
		}
	}
	\end{lstlisting}
	mas veja que podemos fazer a fusão dos laços, 
	\begin{lstlisting}
        int i = 0;
	    for (i = 0; i < sistema.numFuncoes - 1; i++) {
		    sprintf(var,"x%d", i+2);
		    (*diagonais)[0][i] = evaluator_derivative(sistema.funcoes[i], var);
		    
		    sprintf(var,"x%d", i+1);
		    (*diagonais)[1][i] = evaluator_derivative(sistema.funcoes[i], var);
		    (*diagonais)[2][i] = evaluator_derivative(sistema.funcoes[i+1], var);
	    }
	    sprintf(var,"x%d", i+1);
	    (*diagonais)[1][i] = evaluator_derivative(sistema.funcoes[i], var);
	\end{lstlisting}
	e agora aplicamos \textit{Unroll} \& \textit{Jam},
	\begin{lstlisting}
	void geraMatTridiagonalDerivParcial(void ****diagonais, SNL sistema) {
		int i = 0;
		int limit = (sistema.numFuncoes - 1) - (sistema.numFuncoes - 1) % 4; 
		for (i = 0; i < limit; i += 4) {
			sprintf(var,"x%d", i+2);
			(*diagonais)[0][i] = evaluator_derivative(sistema.funcoes[i], var);
			sprintf(var,"x%d", i+1);
			(*diagonais)[1][i] = evaluator_derivative(sistema.funcoes[i], var);
			(*diagonais)[2][i] = evaluator_derivative(sistema.funcoes[i+1], var);
			
			sprintf(var,"x%d", i+3);
			(*diagonais)[0][i+1] = evaluator_derivative(sistema.funcoes[i+1], var);
			sprintf(var,"x%d", i+2);
			(*diagonais)[1][i+1] = evaluator_derivative(sistema.funcoes[i+1], var);
			(*diagonais)[2][i+1] = evaluator_derivative(sistema.funcoes[i+2], var);
			
			sprintf(var,"x%d", i+4);
			(*diagonais)[0][i+2] = evaluator_derivative(sistema.funcoes[i+2], var);
			sprintf(var,"x%d", i+3);
			(*diagonais)[1][i+2] = evaluator_derivative(sistema.funcoes[i+2], var);
			(*diagonais)[2][i+2] = evaluator_derivative(sistema.funcoes[i+3], var);
			
			sprintf(var,"x%d", i+5);
			(*diagonais)[0][i+3] = evaluator_derivative(sistema.funcoes[i+3], var);
			sprintf(var,"x%d", i+4);
			(*diagonais)[1][i+3] = evaluator_derivative(sistema.funcoes[i+3], var);
			(*diagonais)[2][i+3] = evaluator_derivative(sistema.funcoes[i+4], var);
		}
		
		// Resíduos
		for (i = limit; i < sistema.numFuncoes - 1; i++) {
			sprintf(var,"x%d", i+2);
			(*diagonais)[0][i] = evaluator_derivative(sistema.funcoes[i], var);
			sprintf(var,"x%d", i+1);
			(*diagonais)[1][i] = evaluator_derivative(sistema.funcoes[i], var);
			(*diagonais)[2][i] = evaluator_derivative(sistema.funcoes[i+1], var);
		}
		sprintf(var,"x%d", i+1);
		(*diagonais)[1][i] = evaluator_derivative(sistema.funcoes[i], var);
	}
	\end{lstlisting}
	chegando na noss função final. Note que a diagonal superior está no indice $ 0 $, principal no $ 1 $ e inferior no $ 2 $.
\end{document}